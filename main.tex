% Using mnras_template.tex
%
% LaTeX template for creating an MNRAS paper
%
% v3.0 released 14 May 2015
% (version numbers match those of mnras.cls)
%
% Copyright (C) Royal Astronomical Society 2015
% Authors:
% Keith T. Smith (Royal Astronomical Society)

%%%%%%%%%%%%%%%%%%%%%%%%%%%%%%%%%%%%%%%%%%%%%%%%%%
\documentclass[fleqn,usenatbib]{mnras}
\usepackage[T1]{fontenc}

% Allow "Thomas van Noord" and "Simon de Laguarde" and alike to be sorted by "N" and "L" etc. in the bibliography.
% Write the name in the bibliography as "\VAN{Noord}{Van}{van} Noord, Thomas"
\DeclareRobustCommand{\VAN}[3]{#2}
\let\VANthebibliography\thebibliography
\def\thebibliography{\DeclareRobustCommand{\VAN}[3]{##3}\VANthebibliography}

%%%%% AUTHORS - PLACE YOUR OWN PACKAGES HERE %%%%%
\usepackage{graphicx}	% Including figure files
\usepackage{amsmath}	% Advanced maths commands
\usepackage{amssymb}	% Extra maths symbols

%%%%%%%%%%%%%%%%%%%%%%%%%%%%%%%%%%%%%%%%%%%%%%%%%%

%%%%% AUTHORS - PLACE YOUR OWN COMMANDS HERE %%%%%


\newcommand{\vdag}{(v)^\dagger}
\newcommand\latex{La\TeX}
\newcommand{\Msun}{\,{\rm M}$_{\odot}$\,}
\newcommand{\Msunh}{\,{\rm M}$_{\odot}$\,\,\ifmmode h^{-1}\else $h^{-1}$\fi}
\newcommand{\kms}{\,{\rm km s}\ifmmode ^{-1}\,\else $^{-1}$\,\fi}
\newcommand{\Mpch}{\,{\rm Mpc}\,\ifmmode h^{-1}\else $h^{-1}$\fi}
\newcommand{\kpch}{\,{\rm kpc}\,\ifmmode h^{-1}\else $h^{-1}$\fi}
\newcommand{\kpc}{\,{\rm kpc}\,}

%%%%%%%%%%%%%%%%%%%%%%%%%%%%%%%%%%%%%%%%%%%%%%%%%%

%%%%%%%%%%%%%%%%%%% TITLE PAGE %%%%%%%%%%%%%%%%%%%

\title[Cosmic Web Graph Complexity]{The complexity of the cosmic web graph}

\author[Torres-Guar\'in et al.]{
D. A. Torres-Guar\'in,$^{1}$\thanks{E-mail:
  da.torresg@uniandes.edu.co} 
X.-D. Li$^{2}$
and J. E. Forero-Romero$^{1}$\thanks{E-mail:
  je.forero@uniandes.edu.co} 
\\
% List of institutions
$^{1}$Departamento de F\'isica, Universidad de los Andes, Cra. 1
No. 18A-10 Edificio Ip, CP 111711, Bogot\'a, Colombia\\ 
$^{2}$School of Physics and Astronomy, Sun Yat-Sen University,
Guangzhou 510297, P.R.China\\ 
}

% These dates will be filled out by the publisher
\date{Accepted XXX. Received YYY; in original form ZZZ}

% Enter the current year, for the copyright statements etc.
\pubyear{2020}

% Don't change these lines
\begin{document}
\label{firstpage}
\pagerange{\pageref{firstpage}--\pageref{lastpage}}
\maketitle

% Abstract of the paper
\begin{abstract}
We explore complexity as a measure of structure for the cosmic web. 
We find a suitable complexity definition by using the $\beta$-skeleton graph and
the probability $P(n)$ of a node having $n$ connections. 
We compute complexity as the product between the Shannon entropy
$S(P)$ and the Jensen-Shannon divergence  between $P(n)$ and the number
probability of a random set of points $P_{ran}(n)$. 
We find that the complexity is on the order of $10^{-1}$ bits$^2$,
with a non-monotonous dependence with the $\beta$ parameter, showing
a minimum around $\beta\approx3.5$.
We also study the influence on the  complexity of six factors:  cosmic
variance, geometry, Redshift Space Distortions (RSD), redshift
evolution, cosmological parameters and tracer number density. 
We measure the strongest influence produced by the tracer number density.
Among the cosmological parameters considered $\sigma_{8}$ had the
biggest correlation with complexity.  

\end{abstract}
\begin{keywords}
cosmology: large-scale structure of Universe -- methods: data analysis.
\end{keywords}

%%%%%%%%%%%%%%%%%%%%%%%%%%%%%%%%%%%%%%%%%%%%%%%%%%

%%%%%%%%%%%%%%%%% BODY OF PAPER %%%%%%%%%%%%%%%%%%
\section{Introduction}
The cosmic web is the pattern of the matter distribution in the
universe on the largest scales (about a hundred megaparsecs). 
It is generally agreed that this structure is the result of the
perturbations in the density field at the beginning of the universe
driven entirely by the gravitational force \cite{cosmic_web}. 
This makes the structure of cosmic web an interesting and important object
to study, because it helps us understand the formation and evolution
of galaxies \cite{Cautun_2014}.

There are several approaches to describe the structure of the cosmic
web. One of the most commonly used is based on the distribution of
matter in the universe being a scalar field that varies point to
point. In this framework the main goal is to identify low and high
density structures, such as voids, knots, filaments and sheets
\cite{2018MNRAS.473.1195L}. Another approach treats galaxies and dark
matter halos as punctual objects, hence a discrete distribution of
matter is considered. In order to describe these last distributions
researchers have used geometry and graph theory concepts such as the
Voronoi diagram \cite{voronoi} and the $\beta$-skeleton
\cite{beta_skeleton}.

As to how to describe these graphs, there are several options, such as
the average number of connections, the average length of the edges, or
even the average direction for a direct graph. Another possible
quantity is the probability distribution over the possible numbers of
connections. This distribution tells us how connected are the nodes in
the graph and how homogeneous the graph is.

There is a measure of structure called statistical complexity for
systems in which it is possible to define a probability
distribution. Although there is more than one definition for the
statistical complexity every one of them vanishes for fully certain
and fully random systems. The reason being that neither of such
extremes in randomness have structure \cite{grassberg}
\cite{HUBERMAN1986376}. The authors that have worked on this problem
define the complexity as the product of the entropy $S$ and a term
called disequilibrium $D$ \cite{lopez_comp}. The purpose of the
disequilibrium is to be zero for the fully random distribution
(i.e. the uniform distribution). There has been attempts to find an
appropriate expression for $D$, and some problems that have arisen are
the dependence with entropy and with the size of the system.

In this work we use the $\beta$-skelton graph to study the structure
of the cosmic web. We then apply the definition of statistical
complexity introduced by \cite{sig_com} to the resulting graph to
compute the complexity of the cosmic web. Similar information theory
concepts have been used in other physical subjects like quantum and
statistical mechanics \cite{comp_quantum} and observational
astrophysics \cite{comp_astro}.   

\section{Graphs and Complexity}

\subsection{The Beta-Skeleton Graph}

The Gabriel graph of a set of points is the graph in which two points
are connected whenever there is no other point in the region enclosed
by the circle whose diameter is the distance between the points
(fig. \ref{fig:gabriel}).   
It was introduced by the mathematician K. Ruben Gabriel in 1969. 
The $\beta$-skeleton can be thought of as a generalization of the
Gabriel graph, in which the exclusion region depends on a real
parameter beta (see fig.\ref{fig:bskeleton_area}).   
As $\beta$ increases the exclusion  region gets larger and the graph tends to be
more disconnected.  

\begin{figure}
    \centering
    \includegraphics[width=0.45\textwidth]{gabriel.pdf}
    \caption{The two possible scenarios for the Gabriel graph. On the left side the points $a$ and $b$ are connected since the third point is outside the colored region. On the right side the third point is inside the region and the points $a$ and $b$ are disconnected.}
    \label{fig:gabriel}
\end{figure}
\begin{figure}
    \centering
    \includegraphics[width=0.45\textwidth]{betaskeleton.pdf}
    \caption{Exclusion region in the $\beta$-skeleton for different values of $\beta$. If beta is less than 1 the exclusion region is defined as the intersection between the two circles of radius $d/\beta$ that pass through the points. For $\beta>1$ the exclusion region is the intersection between the two circles with diameter $d\beta$ that pass through one of the points while the other lies on a diameter.}
    \label{fig:bskeleton_area}
\end{figure}
\subsection{Graph Complexity Definition}

\begin{figure}
    \centering
    \includegraphics[width=0.45\textwidth]{svb.pdf}
    \caption{Entropy as a function of $\beta$ for random and simulated points.}
    \label{fig:svb}
\end{figure}

\begin{figure*}
    \centering
    \includegraphics[width=0.45\textwidth]{cvb_viejo.pdf}
    \includegraphics[width=0.45\textwidth]{cvb_inicial.pdf}
    \caption{First definition of complexity as a function of $\beta$
      for random and simulated points.} 
    \label{fig:cvb_viejo} 
\end{figure*}

Once the graph is constructed the probability of a node having $n$
connections can be calculated as $p_n=N_{n}/T$. Where $N_{n}$ is the
number of nodes with n connections and T is the total number of
nodes. The next task is to find a scalar quantity that captures the
structure of the graph. In this paper we have chosen the statistical
complexity to be this quantity. One possible definition of complexity
is found in \cite{lopez_comp}, where it is defined as: 

\begin{equation}
	C(P)=S(P)D(P),
\end{equation}
where $S(P)$ is the Shannon entropy and
\begin{equation}
	D(P)=\sum_{n=1}^{N}\left(p_n - 1/N\right)^{2}.
\end{equation}
The motivation behind this type of definition is that it vanishes
either if P has a well-defined outcome or if it is the uniform
distribution.  Hence this definition is in agreement with the claim
that pure randomness and pure certainty possess no structure.  
However, as it is shown in \cite{sig_com} it is more convenient to
work with another definition of complexity: 
\begin{equation}
    C(P)=\frac{D(P,U)H(P)}{D^{*}}.
    \label{eq:comp_def}
\end{equation}
where $H(P)$ is the normalized Shannon entropy, $D(P,U)$ is the
Jensen-Shannon divergence between $P$ and the uniform distribution
$U$, 
\begin{equation}
	D(P,U)=S\left(\frac{P+U}{2}\right) - \frac{S(P)+S(U)}{2}.
\end{equation}
and $D^{*}$ is a normalization constant,
\begin{equation}
    D^{*}=-\frac{1}{2}\left[\frac{n+1}{n}\log_2(n+1)+\log_2(n)-2\log_2(2n)\right].
\end{equation}
Notice that the purpose of the Jensen-Shannon divergence is to make
the complexity vanish whenever $P$ is the uniform. Hence, by computing
this quantity with another distribution instead, we are changing what
we consider to be a zero complexity distribution (apart from the
delta-like distribution). The reason to change the definition of
complexity once again shall become clear in the following sections. 
\section{Methods}


\subsection{Simulations and Mock Catalogs}
The Abacus project \citep{abacus} is a set of N-body simulations of
dark matter halos. It includes simulations with different box sizes
and cosmological parameters. We use simulations of a cubic box of
$720$\Mpch on each side with $1440^3$ particles. Each particle in the
simulation has mass of $\sim1\times10^{10}$ \Msunh. The cube was
simulated with 20 different initial conditions at standard cosmology
and with 40 different cosmological parameters with the same initial
conditions. 
We build mock catalogs with two different geometries: spheres and
spherical shells. In both cases the radius is $300$ \Mpch. The shells
have an inner radius of $250$\Mpch. For every catalog we construct
their random counterparts by randomizing the angular coordinates and
fixing the radius. We also produced catalogs with and without Redshift
Space Distortion (RSD) effects. To study the effect of redshift
evolution we use spheres extracted from the snapshots at redshifts of
$z=0.1$, $0.3$, $0.5$, $0.7$, $1$ and $1.5$. 




\subsection{Numerical Experiments}

For each mock catalog we construct the $\beta$-skeleton and compute
the probability of a node having $n$ connections $p_n$. From this
distribution we calculate the complexity using equation
\ref{eq:comp_def}.  

First we compare the complexity and entropy of a simulated sphere of
points to that of a random sphere for $\beta$ between 1.0 and 5.0. The
results found in this analysis motivate us to redefine our definition
of complexity. The new equation is still basically \ref{eq:comp_def},
but now the Jensen-Shannon divergence is computed between the
distribution of simulated points and the distribution of random points
(not the uniform distribution). From now on we will use this
definition. 

We calculated the complexity for different values of $\beta$ and
examined the influence of several variables on it. These variables
were: cosmic variance, geometry, RSD, redshift evolution, cosmological
parameters and number density. 







\section{Results}
\label{sec:results}

Figure \ref{fig:cvb_viejo} shows the complexity as a function of
$\beta$ for simulated and random points distributed within a
sphere. There are two relevant things to underline. First, the
complexity is not a continuous function of $\beta$. Second, the
complexity of the random set of points is not zero or even close to
zero compared to the simulated points. We found that the
discontinuities of the complexity correspond to the values of $\beta$
for which the number of possible connections changes. For instance, at
the discontinuity near $\beta$=4.0 the number of possible connections
goes from 6 to 5. The second issue is easily explained since the
random points do not generate a uniform distribution for the
connections. Figures \ref{fig:svb} and \ref{fig:cvs_viejo} compare
this definition of complexity to the Shannon entropy. We can see that
the entropy, unlike the complexity, distinguishes between random and
simulated points.  

Therefore, it is convenient to change the definition of complexity, 
\begin{equation}
    C(P)=\frac{D(P,P_{random})H(P)}{D^{*}},
    \label{eq:comp_def2}
\end{equation}
where $P_{random}$ is the distribution of number of connections for
the random points. 

This new complexity, by construction, solves the problem of a random
distribution having a complexity different from zero. Moreover, as
shown in figure \ref{fig:cvb}, it turns out to be a smooth function of
$\beta$ so fewer values of $\beta$ are needed. The complexity follows
a decreasing tendency until it reaches its minimum value near
$\beta$=3.3. The order of magnitude is $10^{-1}$ bits$^{2}$. 

\subsection{Cosmic Variance}

We compare the complexity of 20 different spheres of standard
cosmology built at $z=0.1$ without Redshift Space Distortions. Panel
$a)$ in figure \ref{fig:4graf} shows the complexity of the 20 spheres
with respect to one of them. From this plot it is easy to see that for
$\beta$ 3.5 the complexity values are very similar. In fact, at this
point the standard deviation is minimum (3.0$\times10^{-4}$
bits$^{2}$) and for $\beta$=1.0 it is maximum (3.1$\times10^{-3}$
bits$^{2}$).  



\subsection{Geometry}

We compare the complexity of 20 different spheres with that of 20
different spherical shells for standard cosmology,$z=0.1$, and no
RSD. Panel $b)$ in figure \ref{fig:4graf} shows the mean complexity
for the spheres and shells, as well as the standard deviation. For
$\beta\leq3.0$ the complexity of the spheres is larger than that of
the shells and are approximately equal afterwards. The difference
between spheres and shells is of order $10^{-2}$ bits$^{2}$. 




\subsection{Redshift Space Distortions}

We compare the complexity of 20 different spheres with and without RSD
for standard cosmology and $z=0.1$. Panel $c)$ in figure
\ref{fig:4graf} shows the mean complexity for both cases and the
standard deviation. For $\beta\leq3.0$ the spheres with RSD have a
bigger complexity than those without RSD and for $\beta\geq3.0$ the
situation is the opposite. The differences between both cases is of
order $10^{-2}$ bits$^2$. 


\subsection{Redshift Evolution}
We compute the complexity for six values of redshift $z=$0.1, 0.3,
0.5, 0.7, 1.0, 15 without considering RSD. Panel $d)$ in figure
\ref{fig:4graf} shows the evolution of complexity with $z$ for
$\beta=$1.0, 1.5, 2.0, 2.5, 3.0. For each value of $\beta$ the
complexity is taken with respect to the value at $z=0.1$. We see that
for most values of $\beta$ there is an increasing tendency as $z$ gets
larger. This is not the case for $\beta$=1.0, which falls rapidly
between $z=1.0$ and $z=1.5$. The differences in complexity range from
$10^{-2}$ to $10^{-3}$ bits$^2$. 

\subsection{Cosmological Parameters}
We used the 40 different cosmologies available in the Abacus project
to measure the influence on the complexity of the cosmological
parameters $H_0$, $\Lambda$, $\Omega_{M}$, $n_s$, $\sigma_8$ and
w$_0$. We calculated the complexity at $z=0.1$ without RSD and for
$\beta$=1.0, 2.0, 3.5 and 5.0.  The Spearman’s rank correlation
coefficient $\rho$ was used as a measure of correlation between $C$
and each one of the cosmological parameters. Figure \ref{fig:rhovb}
shows $\rho$ as a function of $\beta$ for every cosmological parameter
considered.  For most parameters the strongest correlation was found
at $\beta$=5.0 and there was an abrupt change in the sign of $\rho$
from $\beta$=3.5 to 5.0. The exception was $n_{s}$ with the smallest
correlation and variations with $\beta$. Figure \ref{fig:cvsigma8}
shows the complexity for the 40 different values of $\sigma_{8}$ at
$\beta=5.0$. This is the case with the strongest correlation, for
which $\rho=0.511$. 

\subsection{Number densities}
We compare the complexity of one sphere when different percentages of
points are sampled. Figure \ref{fig:cvb_porcentaje} shows $C$ as a
function of $\beta$ for every percentage considered. Although there
are some exceptions, the complexity gets larger as one samples more
and more points. The biggest changes are found at $\beta$=1.0 (maximum
complexity) and the smallest ones at $\beta=3.5$ (minimum
complexity). This is the most influential variable among the ones
considered in this work as it has differences of the same order as the
complexity ($10^{-1}$ bits$^2$). 






\section{Conclusions}

In this paper we study several definitions of complexity to measure the structure of the cosmic web. We first construct the $\beta$-skeleton graph from N-body simulations and then compute the probability of a node having $n$ connections as $p_{n}=N_{n}/T$. It is on this distribution that our calculations are performed. We explore a definition proportional to the Shannon entropy and the Jensen-Shannon divergence between $P$ and the uniform distribution (eq. \ref{eq:comp_def}). This definition fails at assigning random distributions of points a complexity value different from zero and being a discontinuous function of $\beta$. We found that both issues can be solved by computing the Jensen-Shannon divergence with the probability distribution of a random set of points instead. This is due to the fact that random points do not have a uniform distribution of number of connections. 

We then study how this definition of complexity is affected by six
factors: cosmic variance, geometry, Redshift Space Distortions (RSD),
redshift evolution, cosmological parameters and number density. Cosmic
variance turned out to be the least influential factor, with
differences ranging from $10^{-4}$ to $10^{-3}$ bits$^2$. This means
that the cosmic web is very homogeneous as far as complexity is
concerned. The next factors are: cosmological parameters with changes
of order $10^{-3}$ bits$^2$; redshift evolution (10$^{-3}$-10$^{-2}$);
RSD (10$^{-2}$) and geometry (10$^{-2}$). The most influential factor
is the number density with changes of the same order as the complexity
($10^{-1}$ bits$^2$) and a clear distinction between low and high
number density.  

\section*{Acknowledgements}
XDL acknowledges the support from NSFC grant (No. 11803094).

\bibliographystyle{mnras}
\bibliography{references}



%\begin{figure}
%    \centering
%    \includegraphics[width=0.45\textwidth]{cvs_viejo.pdf}
%    \caption{First definition of complexity versus entropy for random and simulated points.}
%    \label{fig:cvs_viejo}
%\end{figure}

\begin{figure}
    \centering
    \includegraphics[width=0.45\textwidth]{cvs.pdf}
    \caption{Complexity versus entropy.}
    \label{fig:cvs}
\end{figure}
\begin{figure*}
    \centering
    \includegraphics[width=0.4\textwidth]{varianza.pdf}
    \includegraphics[width=0.4\textwidth]{geometria.pdf}
    \includegraphics[width=0.4\textwidth]{rsd.pdf}
    \includegraphics[width=0.4\textwidth]{rsd_evolution.pdf}
    \caption{a) Complexity as a function of $\beta$ for 20 spheres centered at different points. We plot the difference with respect to the complexity of one of the spheres $C_0$. b) Mean complexity as a function of $\beta$ for 20 spheres and shells centered at different points. The vertical lines represent the standard deviation. c) Mean complexity as a function of $\beta$ for 20 spheres, both considering redshift space distortions and not. The vertical lines represent the standard deviation. d) Complexity with respect to $z=0.1$ as a function of $z$ for 6 different values of $\beta$.}
    \label{fig:4graf}
\end{figure*}
\begin{figure}
    \centering
    \includegraphics[width=0.45\textwidth]{rhovb.pdf}
\caption{Spearman's rank correlation coefficient between the statistical complexity and the cosmological parameters as a function of $\beta$.}
    \label{fig:rhovb}
\end{figure}
\begin{figure}
    \centering
    \includegraphics[width=0.45\textwidth]{cvsigma8.pdf}
    \caption{Complexity as a function of $\sigma_{8}$ for $\beta=5.0$. The Spearman's correlation coefficient is 0.511, the largest for the cosmological parameters considered.}
    \label{fig:cvsigma8}
\end{figure}

\begin{figure}
    \centering
    \includegraphics[width=0.5\textwidth]{cvb_porc.pdf}
    \caption{Complexity as a function of $\beta$ for several percentages of sampled points.}
    \label{fig:cvb_porcentaje}
\end{figure}
\end{document}
